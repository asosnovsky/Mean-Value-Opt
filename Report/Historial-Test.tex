\documentclass[]{article}
\usepackage{lmodern}
\usepackage{amssymb,amsmath}
\usepackage{ifxetex,ifluatex}
\usepackage{fixltx2e} % provides \textsubscript
\ifnum 0\ifxetex 1\fi\ifluatex 1\fi=0 % if pdftex
  \usepackage[T1]{fontenc}
  \usepackage[utf8]{inputenc}
\else % if luatex or xelatex
  \ifxetex
    \usepackage{mathspec}
    \usepackage{xltxtra,xunicode}
  \else
    \usepackage{fontspec}
  \fi
  \defaultfontfeatures{Mapping=tex-text,Scale=MatchLowercase}
  \newcommand{\euro}{€}
\fi
% use upquote if available, for straight quotes in verbatim environments
\IfFileExists{upquote.sty}{\usepackage{upquote}}{}
% use microtype if available
\IfFileExists{microtype.sty}{%
\usepackage{microtype}
\UseMicrotypeSet[protrusion]{basicmath} % disable protrusion for tt fonts
}{}
\usepackage[margin=1in]{geometry}
\usepackage{graphicx}
\makeatletter
\def\maxwidth{\ifdim\Gin@nat@width>\linewidth\linewidth\else\Gin@nat@width\fi}
\def\maxheight{\ifdim\Gin@nat@height>\textheight\textheight\else\Gin@nat@height\fi}
\makeatother
% Scale images if necessary, so that they will not overflow the page
% margins by default, and it is still possible to overwrite the defaults
% using explicit options in \includegraphics[width, height, ...]{}
\setkeys{Gin}{width=\maxwidth,height=\maxheight,keepaspectratio}
\ifxetex
  \usepackage[setpagesize=false, % page size defined by xetex
              unicode=false, % unicode breaks when used with xetex
              xetex]{hyperref}
\else
  \usepackage[unicode=true]{hyperref}
\fi
\hypersetup{breaklinks=true,
            bookmarks=true,
            pdfauthor={Ari Sosnovsky},
            pdftitle={Mean Variance Optimization - Historical Test},
            colorlinks=true,
            citecolor=blue,
            urlcolor=blue,
            linkcolor=magenta,
            pdfborder={0 0 0}}
\urlstyle{same}  % don't use monospace font for urls
\setlength{\parindent}{0pt}
\setlength{\parskip}{6pt plus 2pt minus 1pt}
\setlength{\emergencystretch}{3em}  % prevent overfull lines
\setcounter{secnumdepth}{0}

%%% Use protect on footnotes to avoid problems with footnotes in titles
\let\rmarkdownfootnote\footnote%
\def\footnote{\protect\rmarkdownfootnote}

%%% Change title format to be more compact
\usepackage{titling}

% Create subtitle command for use in maketitle
\newcommand{\subtitle}[1]{
  \posttitle{
    \begin{center}\large#1\end{center}
    }
}

\setlength{\droptitle}{-2em}
  \title{Mean Variance Optimization - Historical Test}
  \pretitle{\vspace{\droptitle}\centering\huge}
  \posttitle{\par}
  \author{Ari Sosnovsky}
  \preauthor{\centering\large\emph}
  \postauthor{\par}
  \predate{\centering\large\emph}
  \postdate{\par}
  \date{November 30, 2015}



\begin{document}

\maketitle


After the optimization process we run the model on the 2008-2012 period.
The method involved partitioning the data into years, and then estimated
the optimal weights of each year. Then we took those weights and applied
them to future prices.

That is, we estimated the weights \(W^T_j = [w_1,...,w_{10}]\), with
their respected indexes \(P_j = [...]\) for year \(j\). Then applied
them to the future market prices, where if \(M_{j+1}\) is a matrix of
market prices of all stock prices at year \(j+1\), then our future
prices in 2009 are \(w^T_{2008}M_{2009}(:,P_{2008})\).

The results are as follows:
\includegraphics{Historial-Test_files/figure-latex/plot-1.pdf}

As you can see, each implimentation follows a similar trend. We see a
clear bad optimization period for the 2009-2010 years. This is due to
the fact that the weights in this time are optimized assuming the trend
of the 2008 year will continue to the next year. However, 2008 had a
particularly bad period for the economy, which did not last after the
year ended. Hence, the period itself provided the wrong idea as for what
will be seen in the future.

The same can be said about the optimization period of 2012-2013, this
period was optimized using prices from 2011-2012, which was the time
when Greece defaulted, which sent a major shock to the economy.

These sort of examples expose the inefficiency of MVP when it is applied
to large datasets.

\end{document}
